\documentclass[
fontsize=12pt,
paper=a4,
ngerman,
DIV=calc,
oneside, %twoside
numbers=enddot,
headsepline,
listof=totoc,
bibliography=totoc,
index=totoc,
pointlessnumbers,
draft=false, 
parskip=full-,
BCOR=5mm %TODO
]{scrbook} %scrreprt, scrbook

\clubpenalty = 10000
\widowpenalty = 10000 
\displaywidowpenalty = 10000

\usepackage[hyphens]{url}


\usepackage[ngerman]{babel}
\usepackage[T1]{fontenc}

%\usepackage{lmodern}       %Schriftart 
\usepackage{pslatex}

\usepackage{textcomp} % Euro-Zeichen etc.
\usepackage{microtype}
\usepackage[utf8]{inputenc}
\usepackage[paper=a4paper,left=35mm,right=25mm,top=20mm,bottom=20mm,includeheadfoot]{geometry}
\setlength{\footskip}{2cm} 

\usepackage[bookmarksdepth=subsection,colorlinks=false,pdfborder={0 0 0},plainpages=false]{hyperref}
%fuer hyperlinks im pdf-format
%PDF-META
\hypersetup{
 pdftitle={\pdftitle},
    pdfsubject={\pdfsubject},
    pdfauthor={\pdfauthor},
    pdfkeywords={\pdfkeywords}
  }


\usepackage{csquotes}


%\usepackage[round]{natbib}


%(\RequirePackage[ngerman=ngerman-x-latest]{hyphsubst}) Silbentrennung ggf

%Zitieren
\usepackage[
backend=biber,
%style=alphabetic,    % Zitierstil
bibstyle=authoryear,
citestyle=authoryear,
isbn=false,                % ISBN nicht anzeigen, gleiches geht mit nahezu allen anderen Feldern
url=false,
doi=false,
pagetracker=true,          % ebd. bei wiederholten Angaben (false=ausgeschaltet, page=Seite, spread=Doppelseite, true=automatisch)
maxbibnames=10,            % maximale Namen, die im Literaturverzeichnis angezeigt werden 
maxcitenames=2,            % maximale Namen, die im Text angezeigt werden, ab 3 wird u.a. nach den ersten Autor angezeigt
autocite=inline,           % regelt Aussehen für \autocite (inline=\parancite)
block=space,               % kleiner horizontaler Platz zwischen den Feldern
backref=false,              % Seiten anzeigen, auf denen die Referenz vorkommt
backrefstyle=three+,       % fasst Seiten zusammen, z.B. S. 2f, 6ff, 7-10
date=short,                % Datumsformat
%natbib=true,
bibencoding=utf8,
%firstinits=true, %depr given
giveninits=true,
%uniquename=init,
uniquename=false,
dashed=false,
  clearlang=true
]{biblatex}

\AtEveryBibitem{%
  \clearlist{language}%
}

%%%%%%%%%%%%%%%%%%%%%%%%%%
%% Abkürzungsverzeichnis
%%%%%%%%%%%%%%%%%%%%%%%%%%
\usepackage[intoc,german]{nomencl}
\renewcommand{\nomname}{Abkürzungsverzeichnis}
\let\abk\nomenclature
\setlength{\nomlabelwidth}{.20\hsize}
\renewcommand{\nomlabel}[1]{#1 \dotfill}
\setlength{\nomitemsep}{-\parsep}
\makenomenclature

\usepackage{verbatim}

\usepackage{setspace}

%\usepackage{scrpage2}
%\usepackage[automark,footsepline,plainfootsepline,headsepline]{scrpage2} 
\usepackage[automark,headsepline]{scrpage2} 

%\automark[section]{chapter} 
%\usepackage{scrpage2} 



\renewcommand*{\finalnamedelim}{%
  \ifnumgreater{\value{liststop}}{2}{\finalandcomma}{}%
  \addspace\&\space}

\DefineBibliographyStrings{ngerman}{ 
   andothers = {{et\,al\adddot}},             
} 
 
 \renewcommand*{\nameyeardelim}{\addcomma\space}

\usepackage{xargs}                      % Use more than one optional parameter in a new commands

\usepackage[colorinlistoftodos,prependcaption,textsize=tiny]{todonotes}
\newcommandx{\unsure}[2][1=]{\todo[linecolor=red,backgroundcolor=red!25,bordercolor=red,#1]{#2}}
\newcommandx{\change}[2][1=]{\todo[linecolor=blue,backgroundcolor=blue!25,bordercolor=blue,#1]{#2}}
\newcommandx{\info}[2][1=]{\todo[linecolor=green,backgroundcolor=green!25,bordercolor=green,#1]{#2}}
\newcommandx{\improvement}[2][1=]{\todo[linecolor=purple,backgroundcolor=purple!25,bordercolor=purple,#1]{#2}}

\usepackage{array}

\newcolumntype{P}[1]{>{\hspace{0pt}}p{#1}}
\usepackage{rotating}



%Für hübsche Zitate
\usepackage{epigraph}

% \epigraphsize{\small}% Default
\setlength\epigraphwidth{8cm}
\setlength\epigraphrule{0pt}

\usepackage{etoolbox}

\makeatletter
\patchcmd{\epigraph}{\@epitext{#1}}{\itshape\@epitext{#1}}{}{}
\makeatother

\usepackage{booktabs}



%%%%Testweise
